\let\negmedspace\undefined
\let\negthickspace\undefined
\documentclass[journal,12pt,twocolumn]{IEEEtran}
\usepackage{cite}
\usepackage{amsmath,amssymb,amsfonts,amsthm}
\usepackage{algorithmic}
\usepackage{graphicx}
\usepackage{textcomp}
\usepackage{xcolor}
\usepackage{txfonts}
\usepackage{listings}
\usepackage{enumitem}
\usepackage{mathtools}
\usepackage{gensymb}
\usepackage{comment}
\usepackage[breaklinks=true]{hyperref}
\usepackage{tkz-euclide}
\usepackage{listings}
\usepackage{gvv}                                        
\def\inputGnumericTable{}                                
\usepackage[latin1]{inputenc}                                
\usepackage{color}                                            
\usepackage{array}                                            
\usepackage{longtable}                                      
\usepackage{calc}                                            
\usepackage{multirow}                                        
\usepackage{hhline}                                          
\usepackage{ifthen}                                          
\usepackage{lscape}

\newtheorem{theorem}{Theorem}[section]
\newtheorem{problem}{Problem}
\newtheorem{proposition}{Proposition}[section]
\newtheorem{lemma}{Lemma}[section]
\newtheorem{corollary}[theorem]{Corollary}
\newtheorem{example}{Example}[section]
\newtheorem{definition}[problem]{Definition}
\newcommand{\BEQA}{\begin{eqnarray}}
\newcommand{\EEQA}{\end{eqnarray}}
\newcommand{\define}{\stackrel{\triangle}{=}}
\theoremstyle{remark}
\newtheorem{rem}{Remark}
\begin{document}
\bibliographystyle{IEEEtran}
\vspace{3cm}
\title{10.5.3}
\author{EE23BTECH11027 - K RAHUL$^{*}$% <-this % stops a space
}
\maketitle
\newpage
\bigskip
\renewcommand{\thefigure}{\theenumi}
\renewcommand{\thetable}{\theenumi}
\section{Question:}
The first and the last terms of an AP are 17 and 350 respectively. If the common difference
is 9, how many terms are there and what is their sum?
\section{Answer:}
The expression to find the $n^{th}$ term of the Arithmetic Progression is given by
\begin{equation}\label{eq:1}a_n = a_1 + (n-1)d\end{equation}
Where
\begin{itemize}
\item n is the number of terms in the AP
\item $a_n$ is the $n^{th}$ term of the series
\item $a_1$ is the first term of the series
\item d is the common difference of the AP
\end{itemize}
\bigskip
The question asked for the number of terms present in the AP, hence from \ref{eq:1}
\begin{equation}\label{eq:2} n=\frac{a_n-a_1}{d} + 1\end{equation}
From the data presented in the question,
$$ a_n=350 $$
$$ a_1=17 $$
$$ d=9 $$
Thus, plugging the values into equation \ref{eq:2}
$$\fbox{n = 38}$$
Now, the expression for sum of numbers in an Arithmetic Progression is given by
\begin{equation}\label{eq:3} S_n = (\frac{n}{2})(a_1+a_n)\end{equation}
Where $S_n$ is the sum of n terms of AP and the rest of the symbols carry the same meaning as mentioned above.\\
Plugging the values into equation, we get
$$S_n = 6973$$\\\\
Thus, the number of terms present in the Arithmetic Progression is 38 and sum of terms of the Arithmetic Progression is 6973.
\end{document}

