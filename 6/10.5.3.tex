\let\negmedspace\undefined
\let\negthickspace\undefined
\documentclass[journal,12pt,twocolumn]{IEEEtran}
\usepackage{cite}
\usepackage{amsmath,amssymb,amsfonts,amsthm}
\usepackage{algorithmic}
\usepackage{graphicx}
\usepackage{textcomp}
\usepackage{xcolor}
\usepackage{txfonts}
\usepackage{listings}
\usepackage{enumitem}
\usepackage{mathtools}
\usepackage{gensymb}
\usepackage{comment}
\usepackage[breaklinks=true]{hyperref}
\usepackage{tkz-euclide} 
\usepackage{listings}
\usepackage{gvv}                                        
\def\inputGnumericTable{}                                 
\usepackage[latin1]{inputenc}                                
\usepackage{color}                                            
\usepackage{array}                                            
\usepackage{longtable}                                       
\usepackage{calc}                                             
\usepackage{multirow}                                         
\usepackage{hhline}                                           
\usepackage{ifthen}                                           
\usepackage{lscape}

\newtheorem{theorem}{Theorem}[section]
\newtheorem{problem}{Problem}
\newtheorem{proposition}{Proposition}[section]
\newtheorem{lemma}{Lemma}[section]
\newtheorem{corollary}[theorem]{Corollary}
\newtheorem{example}{Example}[section]
\newtheorem{definition}[problem]{Definition}
\newcommand{\BEQA}{\begin{eqnarray}}
\newcommand{\EEQA}{\end{eqnarray}}
\newcommand{\define}{\stackrel{\triangle}{=}}
\theoremstyle{remark}
\newtheorem{rem}{Remark}
\begin{document}
\bibliographystyle{IEEEtran}
\vspace{3cm}
\title{10.5.3}
\author{EE23BTECH11027 - K RAHUL$^{*}$% <-this % stops a space
}
\maketitle
\newpage
\bigskip
\renewcommand{\thefigure}{\theenumi}
\renewcommand{\thetable}{\theenumi}
\section{Question:}
\subsection{Question statement}
The first and the last terms of an AP are 17 and 350 respectively. If the common difference
is 9, how many terms are there and what is their sum?
\subsection{Solution}
\begin{table}[ht]
\input{tables/table_1-1}
\end{table}
\begin{align}
\begin{aligned}
\label{eq:1}x(k) &= (x(0) + kd)u(n)\\
&=(17+9k)u(k)
\end{aligned}
\end{align}
Thus,
\begin{align}{k = 37}\end{align}
If $|Z|>1$, then
\begin{align}\label{8}X(z) = (17-8z^{-1})({(1-z^{-1})}^{-2}\end{align}
\begin{align}y(n) = x(n) \star u(n)\end{align}
\begin{align}\implies Y(z) = X(z)U(z)\end{align}
\begin{align}Y(z) = \frac{(17-8z^{-1})}{(1-z^{-1})^{3}}\end{align}
\bigskip
Using contour integral to find Z transform, we get
\begin{align}
    y(37) &= \frac{1}{2\pi j} \oint _C Y(z)z^{36}dz\\
    &= \frac{1}{2\pi j} \oint _C \frac{(17-8z^{-1})}{(1-z^{-1})^{3}}z^{36}dz
\end{align}
Now, using Cauchy's residual theorem and observing the fact that 3 repeated poles exist at z = 1, 
\begin{align}
    R &= \frac{1}{(k-1)!}\lim_{x \to c}\frac{d^{k-1}}{dz^{k-1}}((z-c)^kf(z))\\
    &= \frac{1}{2!}\lim_{x \to 1}\frac{d^{k-1}}{dz^{k-1}}((z-1)^3\frac{(17-8z^{-1})}{(1-z^{-1})^{3}}z^{36})\\
    &=\frac{1}{2}\lim_{x \to 1}\frac{d^2}{dz^2}(17z^{39} - 8z^{38})\\
    &= 6973
\end{align}
\begin{flushleft}
\begin{figure}[h]
\renewcommand\thefigure{1}
    \caption{Stem Plot of $x(n)$ v/s n}
    \includegraphics[width=0.5\textwidth]{figs/x(n)_plot.png}
    \label{fig:stem-plot}
\end{figure}
\end{flushleft}
\end{document}
